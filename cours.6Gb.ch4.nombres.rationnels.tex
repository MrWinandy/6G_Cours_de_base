\chapter{Nombres rationnels}



Manuel: $\Delta 4^e$, p.47

\section{Rappels et vocabulaire}

La division de 2 par 5 peut être notée sous forme d'une fraction: $\frac{2}{5}$.
Toute fraction représente donc une division qu'on n'effectue pas.
Comme on ne peut pas diviser par 0, le dénominateur ne peut pas être 0:
p.ex. $\frac{1}{0}$ n'existe pas!

$$ \frac{2}{4} = \frac{1}{2} $$

\attention Le trait de la fraction s'écrit toujours à la même hauteur que le signe d'égalité!

\begin{enumerate}[label=$\triangleright$]
 \item Si on divise le N et le D d'une fraction par le même nombre, alors la fraction est \uline{simplifiée} par ce nombre:
 $$ \dfrac{10}{15}=\dfrac{2}{3} $$
  \item \attention Si le numérateur est une somme ou une différence, alors on ne peut pas simplifier!
    $$ \frac{6+2}{4} = \frac{8}{4}=2 $$
    On doit avoir un seul nombre ou un produit (multiplication!):
    $$ \dfrac{8}{2}= \dfrac{2\cdot 4}{2}=\dfrac{4}{1}=4 $$
 
 \item Une fraction est appelée \uline{irréductible} si elle ne peut pas être simplifiée sans obtenir des nombres décimaux dans le N ou D.
 \id{noter des exemples}
 
 \item Si on multiplie le N et le D d'une fraction par le même nombre, alors la fraction est \uline{amplifiée} par ce nombre:
 $$ \dfrac{1}{2,7}=\dfrac{10}{27} $$
 
 \item Chaque entier peut être écrit sous forme fractionnaire:
 $$ 3=\frac{3}{1}; 5 = \dfrac{5}{1}; 13 = \dfrac{13}{1} $$
 
 \item Mais:
 $$ 1=\frac{1}{1} = \dfrac{2}{2} = \dfrac{3}{3} = ... $$

\end{enumerate}


 
\section{Fractions négatives}
$\frac{3}{5} > \frac{1}{5}$ mais $-\frac{3}{5} < -\frac{1}{5}$.
\id{Tracer une droite graduée verticale.}

De même: $\frac{1}{2} > \frac{1}{3}$ mais $-\frac{1}{2} < -\frac{1}{3}$.

\rem{Il est plus facile de classer les fractions si on réfléchit si elles sont comprises entre 0 et -1 ou bien plus petites que -1.}


$$\begin{array}{rlr}
 (-1):(+2) &=-0,5 &\text{écriture fractionnaire:} \frac{-1}{2}=-\frac{1}{2} \\
 (+1):(-2) &=-0,5 &\text{écriture fractionnaire:} \frac{1}{-2}=-\frac{1}{2} \\
 (-1):(-2) &=0,5  &\text{écriture fractionnaire:} \boxed{\frac{-1}{-2}=\frac{1}{2}}
\end{array}$$

Donc: 
$$ \boxed{\frac{-1}{2}=\frac{1}{-2} =-\frac{1}{2}} \quad \text{et} \quad \boxed{\frac{-3}{-5}=+\frac{3}{5}} $$
et en général:
\prop{
Quels que soient les nombres $a$ et $b$ ($b\overset{!}{\neq}0$):
$$ \dfrac{-a}{-b}=\dfrac{a}{b}  \qquad\text{et}\qquad  \dfrac{-a}{b}=\dfrac{a}{-b}=-\dfrac{a}{b} $$
}

\attention $-\frac{2+7}{3} = \frac{-(2+7)}{3}$. Il faut mettre des parenthèses, car le signe se rapporte à tous les termes du numérateur.


%Ex. 38,41,42,44,45,47 p.57 (T4) \datei{Bicher/T4/T4_p.057.jpg} \\
%Ex. 125 p.63 \datei{Bicher/T4/T4_p.063.jpg}



\subsection{Addition/Soustraction de fractions}
\stitle{Exemples}
\begin{enumerate}[label=\alph*)]
  \item $ \frac{1}{2}+\frac{1}{4} = $
  \item $ \frac{1}{3}-\frac{1}{4}= $
  \item $ \frac{3}{4} {\green +} \frac{{\red -}2}{5} =  \frac{15}{20} {\green +} \frac{{\red -}8}{20} = \frac{15{\green +}({\red -}20)}{7} = \frac{-5}{20} = -\frac{1}{4} $ 
  \item $ \frac{5}{20} + \frac{3}{15} +\frac{15}{25}= $
\end{enumerate}



\ret{1. On simplifie les fractions autant que possible avant de continuer, on avec tout autre calcul! \\
2. Addition/Soustraction $\rightarrow$ DC. \\
3. Le résultat doit être une fraction irréductible.
}

\addexo{Calculer.
\begin{multicols}{2}
\begin{enumerate}
  \item $ \dfrac{3}{4}+\dfrac{5}{6}-\dfrac{7}{12}= $
  \item $ \dfrac{23}{8}-\dfrac{5}{3}-\dfrac{4}{12}= $
  \item $ \dfrac{3}{2}+\dfrac{2}{3}-\dfrac{4}{5}= $
  \item $ \dfrac{11}{12}-\dfrac{3}{4}+\dfrac{2}{3}= $
  \item $ \dfrac{2}{3}-\dfrac{3}{4}+\dfrac{5}{2}= $
  \item $ \dfrac{5}{9}-\dfrac{6}{11}+\dfrac{2}{3}= $
\end{enumerate}
\end{multicols}
}{-}
\addexo{Calculer.
\begin{multicols}{2}
\begin{enumerate}
  \item $ \dfrac{3}{4}-\dfrac{1}{6}-\dfrac{1}{9}= $
  \item $ \dfrac{10}{15}+\dfrac{15}{12}-\dfrac{20}{18}= $
  \item $ \dfrac{3}{4}+\dfrac{7}{8}-\dfrac{4}{3}= $
  \item $ \dfrac{9}{10}-\dfrac{2}{15}+\dfrac{3}{5}= $
  \item $ \dfrac{98}{100}-\dfrac{196}{200}= $
  \item $ 1-\dfrac{8}{12}+\dfrac{18}{15}= $
\end{enumerate}
\end{multicols}
}{-}


%\arrow Ex. 60-65 p.58 (T4) \datei{Bicher/T4/T4_p.058.jpg} \\
%\arrow Ex. 108 p.61 \datei{Bicher/T4/T4_p.061.jpg} \\
%\arrow Ex. 114, 122 p.62 \datei{Bicher/T4/T4_p.062.jpg}


\addexo{Effectuer et réduire les expressions suivantes.
\begin{enumerate}
  \item $ \dfrac{3}{8}+\dfrac{35}{14}= $
  \item $ \dfrac{13}{3}-\dfrac{4}{9}= $
  \item $ \dfrac{13}{3}-\dfrac{4}{9}= $
  \item $ 0,625+\dfrac{7}{4}+\dfrac{33}{22}= $
  \item $ \dfrac{1}{2}+0,25+\dfrac{1}{3}-\dfrac{5}{6}= $
  \item $ \dfrac{11}{12}-0,75+\dfrac{10}{15}= $
\end{enumerate}
}{Enoncé: Effectuer et réduire les expressions suivantes.
\begin{enumerate}
  \item $ \dfrac{23}{8} $
  \item $ \dfrac{35}{9} $
  \item $ \dfrac{35}{9} $
  \item $ \dfrac{31}{8} $
  \item $ \dfrac{1}{4} $
  \item $ \dfrac{5}{6} $
\end{enumerate}
}

\addexo{Effectuer et réduire les expressions suivantes.
\begin{enumerate}
  \item $ \dfrac{14}{21}-\left( \dfrac{15}{20}+2,5-\dfrac{5}{12} \right)= $
  \item $ \dfrac{17}{12}-\dfrac{5^2}{18}= $
  \item $ \dfrac{10}{15}-\dfrac{15}{12}-\dfrac{20}{18}= $
  \item $ -\dfrac{11}{21}-\dfrac{18}{35}= $
  \item $ -\dfrac{66}{39}+\dfrac{4^3}{26}-\left(\dfrac{72}{52}-\dfrac{30}{65}\right)= $
  \item $ \dfrac{35}{10}-\left( \dfrac{39}{78}-\dfrac{12}{56} \right)= $
\end{enumerate}
}{Enoncé: Effectuer et réduire les expressions suivantes.
\begin{enumerate}
  \item $ -\dfrac{13}{6} $
  \item $ \dfrac{1}{36} $
  \item $ -\dfrac{61}{36} $
  \item $ -\dfrac{109}{105} $
  \item $ -\dfrac{2}{13} $
  \item $ \dfrac{45}{14} $
\end{enumerate}
}

\addexo{Effectuer et réduire les expressions suivantes.
\begin{enumerate}
  \item $ -\dfrac{21}{56}+\dfrac{32}{88}-\dfrac{17}{34} = $
  \item $ \dfrac{9}{14}-\dfrac{3}{21}= $
  \item $ -\dfrac{3}{5}-\dfrac{4}{7}-\dfrac{-2}{70}= $
  \item $ -\left(\dfrac{36}{96}+\dfrac{20}{55}\right) -\dfrac{22}{44} = $
  \item $ \dfrac{15}{45}-\dfrac{-12}{210}-\left(\dfrac{15}{75}+\dfrac{-9}{-35}\right)= $
  \item $ \dfrac{18}{20}-\dfrac{18}{135}+\dfrac{-18}{30}= $
\end{enumerate}
}{Enoncé: Effectuer et réduire les expressions suivantes.
\begin{enumerate}
  \item $ -\dfrac{45}{88} $
  \item $ \dfrac{1}{2} $
  \item $ -\dfrac{8}{7} $
  \item $ -\dfrac{109}{88} $
  \item $ -\dfrac{1}{15} $
  \item $ \dfrac{1}{6} $
\end{enumerate}
}



\subsection{Multiplication de fractions}
Un quart de deux tiers:
\begin{center}
	\definecolor{qqqqff}{rgb}{0,0,1}
	\definecolor{qqzzqq}{rgb}{0,0.6,0}
	\definecolor{cqcqcq}{rgb}{0.75,0.75,0.75}
	\begin{tikzpicture}[line cap=round,line join=round,>=triangle 45,x=1.0cm,y=1.0cm]
		\draw [color=cqcqcq,dash pattern=on 1pt off 1pt, xstep=1.0cm,ystep=1.0cm] (0.76,0.75) grid (5.31,4.38);
		\clip(0.76,0.75) rectangle (5.31,4.38);
		\fill[color=qqzzqq,fill=qqzzqq,fill opacity=0.1] (1,4) -- (5,4) -- (5,2) -- (1,2) -- cycle;
		\fill[color=qqqqff,fill=qqqqff,fill opacity=0.1] (1,2) -- (2,2) -- (2,4) -- (1,4) -- cycle;
		\draw (1,1)-- (5,1);
		\draw (5,1)-- (5,4);
		\draw (5,4)-- (1,4);
		\draw (1,4)-- (1,1);
		\draw [color=qqzzqq] (1,4)-- (5,4);
		\draw [color=qqzzqq] (5,4)-- (5,2);
		\draw [color=qqzzqq] (5,2)-- (1,2);
		\draw [color=qqzzqq] (1,2)-- (1,4);
		\draw [color=qqqqff] (1,2)-- (2,2);
		\draw [color=qqqqff] (2,2)-- (2,4);
		\draw [color=qqqqff] (2,4)-- (1,4);
		\draw [color=qqqqff] (1,4)-- (1,2);
	\end{tikzpicture}
\end{center}

$$  \dfrac{1}{4} \cdot \dfrac{2}{3} = \dfrac{1}{6} $$

\rem{On n'a pas besoin d'un DC!
}

%\arrow Ex. 68-77 p.59 (T4) \datei{Bicher/T4/T4_p.059.jpg}

Pour multiplier deux fractions, on:
\begin{enumerate}
	\item note d'abord le signe du produit,
	\item ensuite on simplifie chaque fois un N et un D.
\end{enumerate}

\subsection{Nombre inverse et division de fractions}
\id{pour 6Gb\,: inverse sans base négative.
}

Exemples:
\begin{multicols}{4}
\begin{enumerate}[label=\alph*)]
  \item $ 4 : \ldots = 2$
  \item $ 4 \cdot \ldots = 2 $
  \item $ 6 : \ldots = 1 $
  \item $ 6 \cdot \ldots = 1 $
  \item $ 8 : \ldots = 2 $
  \item $ 8 \cdot \ldots = 2 $
  \item $ 12 : \ldots = 4 $
  \item $ 12 \cdot \ldots = 4 $
\end{enumerate}
\end{multicols}

\prop{Diviser par un nombre $n$ revient à multiplier par son inverse $\frac{1}{n}$.
}

On appelle $\frac{1}{4}$ l'inverse de $4$ et $4$ l'inverse de $\frac{1}{4}$. \\
De même: $-\dfrac{1}{22}$ est l'inverse de $-22$.

\defn{\textbf{Deux nombres} sont \textbf{inverses} l'un de l'autre $\Leftrightarrow$ leur produit est égal à 1.
}

Exemples\,: 1; $\frac{1}{3}$; $-3$ $\frac{2}{3}$; $-\frac{3}{7}$.


%\arrow Ex. 78-83 p.59 (T4) \datei{Bicher/T4/T4_p.059.jpg} \\
%\arrow Ex. 84-98 p.60 \datei{Bicher/T4/T4_p.060.jpg}

Comme $0,25 = \frac{1}{4}$, on peut multiplier par 4 au lieu de diviser par 0,25. \\
($1:0,25 =4$ et donc $5:0,25 = 5  \cdot 4 = 20$.)

